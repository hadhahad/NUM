\documentclass[11pt,american,czech]{article}
\usepackage[T1]{fontenc}
\usepackage[utf8]{inputenc}
\usepackage[a4paper]{geometry}
\geometry{verbose,tmargin=0.2cm,bmargin=0.2cm,lmargin=1.5cm,rmargin=2cm,headheight=0.8cm,headsep=1cm,footskip=0.5cm}
\setcounter{secnumdepth}{3}
\usepackage{url}
\usepackage{amsmath}
\usepackage{amsthm}
\usepackage{amssymb}
\usepackage{graphicx}
\usepackage{setspace}

\usepackage{enumerate} %roman enumiration

\usepackage{threeparttable}
\usepackage{array}

%% C++ code
\usepackage{listings}
\usepackage{xcolor}
\lstset { %
	language=C++,
	backgroundcolor=\color{black!3}, % set backgroundcolor
	basicstyle=\footnotesize,% basic font setting
}

%% Use Times New Roman font for text and Belleek font for math
%% Please make sure that the 'esint' package is turned off in the
%% 'Math options' page.
\usepackage[varg]{txfonts}


%% Indent even the first paragraph in each section
\usepackage{indentfirst}

% completely avoid orphans (first lines of a new paragraph on the bottom of a page)
\clubpenalty=9500

% completely avoid widows (last lines of paragraph on a new page)
\widowpenalty=9500

% disable hyphenation of acronyms
\hyphenation{CDFA HARDI HiPPIES IKEM InterTrack MEGIDDO MIMD MPFA DICOM ASCLEPIOS MedInria}

%%---------------------------------------------------------------------

%% Print out all vectors in bold type instead of printing an arrow above them
%%\renewcommand{\vec}[1]{\boldsymbol{#1}}

% Replace standard \cite by the parenthetical variant \citep
%\renewcommand{\cite}{\citep}

\makeatother
\pagestyle{empty} %turns off page numbering

\usepackage{babel}
\begin{document}
\selectlanguage{czech}
\def\documentdate{24. dubna 2017}
\begin{flushright}
\textbf{	NUM2cv \\
	Vladislav Belov}
\end{flushright}

Řešíme numericky Riccatiho rovnici~(\ref{uloha}) na intervalu $[0.25, 0.45]$ Rungeovými-Kuttovými metodami. Známe její analytické řešení: $u(t)=\big(\frac{1}{\sqrt{2}t}\tan{(\sqrt{2}(c-\frac{1}{t}))}-\frac{1}{2t}\big)e^{t}$, kde klademe $c=1$.

\begin{equation} \label{uloha}
	\begin{split}
		&\dot{u}(t)=t^{-4}e^{t}+u(t)+2e^{-t}u^{2}(t) = f\big(t, u\big), \\
		&u(0.25)=-31,1844.
	\end{split}
\end{equation}

\noindent
Označíme $\tau=$ \textit{integrationTimeStep}, $u_{0}=u(t_{0})$. Použijeme  metody:
\begin{enumerate}[I.]
	\item Euler:
		\begin{equation*}
			\begin{split}
	&k_{1}(\tau)=\tau\cdot f\big(t_{0}, u_{0}\big), \\
	&u(t_0+\tau)=u(t) + k_{1}(\tau).
			\end{split}
		\end{equation*}
	\item Runge-Kutta 2. řádu: \label{RKII}
		\begin{equation*}
			\begin{split}
				&k_{1}(\tau)=\tau\cdot f\big(t_{0}, u_{0}\big), \\
				&k_{2}(\tau)=\tau\cdot f\big(t_{0}+\tau, u_{0}+k_{1}(\tau)\big), \\
				&u(t_0+\tau)=u(t) + \tfrac{1}{2}k_{1}(\tau)+\tfrac{1}{2}k_{2}(\tau).
			\end{split}
		\end{equation*}
	\item Runge–Kutta–Merson:
		\begin{equation*}
			\begin{split}
		&k_{1}(\tau) = \tau\cdot f(t_{0},u_{0}),\\
		&k_{2}(\tau) = \tau\cdot f \big( t_{0} + \tfrac13 \tau, u_{0} + \tfrac13 k_1 \big),\\
		&k_{3}(\tau) = \tau\cdot f \big( t_{0} + \tfrac13 \tau, u_{0} + \tfrac16 k_1 + \tfrac16 k_2 \big),\\
		&k_{4}(\tau) = \tau\cdot f \big( t_{0} + \tfrac12 \tau, u_{0} + \tfrac18 k_1 + \tfrac38 k_3 \big),\\
		&k_{5}(\tau) = \tau\cdot f \big( t_{0} + \tau, u_{0} + \tfrac12 k_1 - \tfrac32 k_3 + 2 k_4 \big),\\
		&u(t_{0}+\tau) = y_0 + \tfrac16 k_1 + \tfrac23 k_4 + \tfrac16 k_5.
			\end{split}
		\end{equation*}
\end{enumerate}

\hrulefill
\vspace{0.1cm} \\

Rozebereme podrobněji metodu ~(\ref{RKII}):
\begin{enumerate}[(a)]
	\item Výsledný graf v porovnání s průběhem analytického řešení:
		\begin{figure}
		
		\end{figure}
	\item Implementace:
		\begin{lstlisting}
		 
 template< typename Problem >
 class RungeKutta : public IntegratorBase
 {
  public:

   RungeKutta( Problem& problem )
   {
	this->k1 = new double[ problem.getDegreesOfFreedom() ];
	this->k2 = new double[ problem.getDegreesOfFreedom() ];
	this->aux = new double[ problem.getDegreesOfFreedom() ];
   }

   bool solve( Problem& problem,
		double* u )
   {
	const int dofs = problem.getDegreesOfFreedom();
	double tau = std::min( this->integrationTimeStep, this->stopTime - this->time );
	long int iteration( 0 );
	while( this->time < this->stopTime )
	{
	  /****
	  * Compute k1
	  */
	  problem.getRightHandSide( this->time, u, k1 );

	  /****
	  * Compute k2
	  */
	  for( int i = 0; i < dofs; i++ )
	    aux[ i ] = u[ i ] + tau * k1[ i ];
	  problem.getRightHandSide( this->time + tau, aux, k2 );

	  /****/
	  for( int i = 0; i < dofs; i++ )
	    u[ i ] += ( tau / 2.0 ) * ( k1[ i ] + k2[ i ] );
	  this->time += tau;
	  iteration++;
	  if( iteration > 100000 )
	  {
	    std::cerr << "The solver has reached the maximum number of iteratoins. " 
	    		<< std::endl;
	    return false;
	  }
	  tau = std::min( tau, this->stopTime - this->time );
	  std::cout << "ITER: " << iteration << " \t tau = " << tau 
	  		<< " \t time= " << time << "         \r " << std::flush;  
	}
	std::cout << std::endl;
	return true;
   }

   ~RungeKutta()
   {
	delete[] k1;
	delete[] k2;
	delete[] aux;
   }

  protected:

   double *k1, *k2, *aux;

 };
		\end{lstlisting}

\end{enumerate}


\end{document}
